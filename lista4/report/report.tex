\documentclass{article}
\usepackage[top=3cm, bottom=3cm, left = 2cm, right = 2cm]{geometry} 
\geometry{a4paper} 
\usepackage[T1]{polski}
\usepackage[utf8]{inputenc}
\usepackage{titling}
\usepackage{caption}
\usepackage{algorithm2e}
\usepackage[parfill]{parskip}
\usepackage{multirow}
\usepackage{amsmath}
\usepackage{graphicx}
\usepackage{pgffor}

\renewcommand\maketitlehooka{\null\mbox{}\vfill}
\renewcommand\maketitlehookd{\vfill\null}

\SetAlgorithmName{Algorytm}{algorytm}{Lista algorytmów}
\SetKwInput{KwData}{Dane}
\SetKwInput{KwResult}{Wynik}
\SetKwComment{Comment}{/* }{ */}

\title{Obliczenia Naukowe}
\author{Karol Janic}
\date{1 grudnia 2023}

\begin{document}

\begin{titlingpage}
    \maketitle
\end{titlingpage}

\tableofcontents

\newpage

\section{Wstęp}
Iterpolacja funkcji za pomocą wielomianu polega na znalezieniu wielomianu $p$ stopnia co najwyżej $n$, 
który będzie spełniał $p(x_i) = y_i$ dla zadanych punktów $(x_0, y_0), (x_1, y_1), \ldots, (x_n, y_n)$, 
gdzie $x_i$ dla $i=0, 1, \ldots, n$ są parami różne. Wielomian $p$ można przedstawić jako kombinację liniową 
$1, x, x^2, \ldots, x^n$. Jednak takie przedstawienie prowadzi do układu równań z macierzą Vandermonde'a 
co prowadzi do zadania źle uwarunkowanego. Bazę wielomianu można zmienić na $1, (x-x_0), (x-x_0) \cdot (x-x_1), \ldots, (x-x_0) \cdot (x-x_1) \cdot \ldots \cdot (x-x_(n-1))$.
Wówczas kolejnymi współczynnikami będą ilorazy różnicowe a taka postać wielomianu nosi nazwę postaci Newtona.

\section{Zadanie 1}
\subsection{Cel}
Celem zadania jest implementacja metody obliczającej ilorazy różnicowe funkcji $f$ dla zadanych węzłów $x_0, x_1, \ldots, x_n$
oraz wartości funkcji w tych węzłach $f(x_0), f(x_1), \ldots, f(x_n).$

\subsection{Rozwiązanie}
Wielomian $p$ postaci:$ \sum_{i=0}^{n} c_i \cdot q_i(x) $, gdzie $c_i$ to ilorazy różnicowe a $q_i(x) = \prod_{i=0}^{i} (x-x_i)$ jest szukanym wielomianem interpolacyjnym, 
gdy $ \sum_{i=0}^{n} c_i \cdot q_i(x_i) = y_i$ dla $i=0, 1, \ldots, n$. Warunek ten można zapisać w postaci układu równań w formie macierzowej:

\[
\begin{bmatrix}
    1 && 0 && \ldots && 0 \\
    1 && q_1(x_i) && \ldots && 0 \\
    \vdots && \vdots && \ddots && \vdots \\
    1 && q_1(x_n) && \ldots && q_n(x_n)
\end{bmatrix}
\begin{bmatrix}
    c_0 \\
    c_1 \\
    \vdots \\
    c_n
\end{bmatrix}
=
\begin{bmatrix}
    y_0 \\
    y_1 \\
    \vdots \\
    y_n
\end{bmatrix}
\]

Rozwiązaniem takiego układu jest:
\[
c_k = f[x_0, x_1, \ldots, x_k]
\] gdzie $f[x_i, x_1, \ldots, x_j]$ jest ilorazem różnicowym opartym na węzłach $x_i, \ldots, x_j$ określonym rekurencyjnie:

\[
f[x_k] = f(x_k), \quad \quad 0 \leq i \leq n
\]
\[
f[x_i, \ldots, x_j] = \frac{f[x_{i+1}, \ldots, x_j] - f[x_i, \ldots, x_{j-1}]}{x_j - x_i}, \quad \quad 0 \leq i < j \leq n
\]
\subsubsection{Opis metody}
Prosta implementacja polegałaby na stworzeniu dwuwymiarowej tablicy, w której zapisywane byłyby kolejne wartości ilorazów. Warto jednak zauważyć, 
że szukane są wyłącznie ilorazy różnicowe, gdzie $x_i = x_0$. Zatem złożoność pamięciową można ograniczyć do $O(1)$ poprzez wyznaczanie ilorazów w miejscu.
Kolejność wyznaczania ilorazów przedstawiona jest poniżej. Łatwo zobaczyć że złożoność obliczeniowa wynosi $O(n^2)$.

\begin{center}
    \begin{tabular}{c|ccccc}
        iteracja & $c_0$ & $c_1$ & $c_2$ & $\cdots$ & $c_n$ \\
        \hline
        1 & $f[x_0]$ & $f[x_1]$ & $f[x_2]$ & $\cdots$ & $f[x_n]$ \\
        \hline
        2 & & $f[x_0, x_1] =$ & $f[x_1, x_2] =$ & & $f[x_{n-1}, x_{n}] =$ \\
        & & $\frac{f[x_1] + f[x_0]}{x_1 - x_0}$ & $\frac{f[x_1] + f[x_2]}{x_2 - x_1}$ & $\cdots$ & $\frac{f[x_{n-1}] + f[x_n]}{x_{n} - x{n-1}}$ \\
        \hline
        3 & & & $f[x_0, x_1, x_2] =$ & & $f[x_{n-2}, x_{n-1}, x_n] =$ \\
        & & & $\frac{f[x_1, x_2] + f[x_0, x_1]}{x_2 - x_0}$ & $\cdots$ & $\frac{f[x_{n-1}, x_{n}] + f[x_{n-2}, x_{n-1}]}{x_n - x_{n-2}}$ \\
        \hline
        $\vdots$ & & & & $\ddots$ & $\vdots$ \\
        \hline
        n & & & & & $f[x_0, x_1, \ldots, x_n] =$ \\
        & & & & & $\frac{f[x_1, \ldots, x_n] + f[x_0, \ldots, x_{n-1}]}{x_n - x_0}$
    \end{tabular}
\end{center}

\newpage

\subsubsection{Pseudokod}
\begin{algorithm}
    \caption{Algorytm wyznaczania ilorazów różnicowych}
    \DontPrintSemicolon
    \KwData{$x, f$}
    \KwResult{$fx$}
    $xLen \gets \texttt{length}(x)$\;
    $fx \gets f$\;
    
    \For{$i = 2$ \KwTo $xLen$}{
        \For{$j = xLen$ \KwSty{downto} $i$}{
            $fx[j] = \frac{fx[j]-fx[j-1]}{x[j] - x[j-i+1]}$\;
        }
    }
\end{algorithm}

\section{Zadanie 2}
\subsection{Cel}
Celem zadania jest implementacja metody wyznaczającej wartość wielomianu zadanego w postaci Newtona dla danego argumentu.

\subsection{Rozwiązanie}
Zauważmy, że wielomian interpolacyjny można zapisać w nastepujący sposób:
\begin{gather}
    \nonumber
    p(x) = \sum_{i=0}^{n} c_i \cdot q_i(x) = c_0 + \sum_{i=1}^{n} c_i \cdot q_i(x) = c_0 + \sum_{i=1}^{n} c_i \cdot (x-x_0) \cdot \ldots \cdot (x-x_i) = \\
    \nonumber
    = c_0 + (c_1 + \sum_{i=2}^{n} c_i \cdot (x-x_1) \cdot \ldots \cdot (x-x_i)) \cdot (x-x_0) = \\
    \nonumber
    = c_0 + (c_1 + (c_2 + \sum_{i=3}^{n} c_i \cdot (x-x_2) \cdot \ldots \cdot (x-x_i)) \cdot (x-x_1) ) \cdot (x-x_0) = \ldots
\end{gather}
Otrzymujemy zależność, która jest bazą do zastosowania uogólnionego schematu Hornera: $p(x) = w_0(x)$, gdzie:
\[
w_n(x) = c_n
\]
\[
w_k(x) = c_k + (x-x_k) \cdot w_{k+1}(x), \quad k=n-1, \ldots, 0
\]

\subsubsection{Opis metody}
Stosując wyżej opisaną zależność można zaimplementować metodę wyznaczania wartości wielomianu.
Złożoność obliczeniowa algorytmu wynosi $O(n)$ a pamięciowa $O(1)$.

\subsubsection{Pseudokod}
\begin{algorithm}
    \caption{Algorytm wyznaczania wartości wielomianu w postaci Newtona}
    \DontPrintSemicolon
    \KwData{$x, fx, t$}
    \KwResult{$ft$}
    $xLen \gets \texttt{length}(x)$\;
    $ft \gets fx[xLen]$\;
    
    \For{$i = xLen-1$ \KwSty{downto} $1$}{
        $ft \gets fx[i] + (t - x[i]) \cdot ft$
    }
\end{algorithm}

\newpage

\section{Zadanie 3}
\subsection{Cel}
Celem zadania jest implementacja metody wyznaczającej postać naturalną wielomianu interpolacyjnego danego w postaci Newtona.

\subsection{Rozwiązanie}
Rozwińmy pierwsze wyrazy zależności otrzymanej w poprzednim zadaniu:
\[
    w_n(x) = c_n
\]
\[
    w_{n-1}(x) = c_{n-1} + (x - x_{n-1}) \cdot w_n(x) = c_{n-1} + (x - x_{n-1}) \cdot c_n = c_n \cdot x - x_{n-1} \cdot c_n + c_{n-1} 
\]
\[
    w_{n-2}(x) = c_{n-2} + (x - x_{n-2}) \cdot w_{n-1} = c_{n-2} + (x - x_{n-2}) \cdot (c_{n-1} + (x - x_{n-1}) \cdot c_n) = 
\]
\[
    = c_n \cdot x^2 + (c_{n-1} - (x_{n-1} + x_{n-2}) \cdot c_n) \cdot x - x_{n-2} \cdot (c_{n-1} - x_{n-1} \cdot c_n) + c_{n-2}
\]
\[
    \vdots
\]
\subsubsection{Opis metody}
Łatwo zauważyć, że początkowe współczynniki przy kolejnych potęgach $x_i$ wynoszą $c_i$, gdzie $0 \leq i \leq n$ a następnie pomniejszane są o $x_i \cdot a_{j+1}$, gdzie $i \leq j < n$ a $a_{j+1}$ jest aktualną wartością współczynnika wyznaczoną wcześniej. Zależność ta prowadzi do algorytmu o złożoności obliczeniowej $O(n^2)$ oraz pamięciowej $O(1)$.


\subsubsection{Pseudokod}
\begin{algorithm}
    \caption{Algorytm wyznaczania postaci naturalnej wielomianu zadanego w postaci Newtona}
    \DontPrintSemicolon
    \KwData{$x, fx$}
    \KwResult{$a$}
    $xLen \gets \texttt{length}(x)$\;
    $a \gets fx$\;
    
    \For{$i = xLen-1$ \KwSty{downto} $1$}{
        \For{$j = i$ \KwTo $xLen-1$}{
            $a[j] \gets a[j] - a[j+1] \cdot x[i]$
        }
    }
\end{algorithm}

\section{Zadanie 4}
\subsection{Cel}
Celem zadania jest implementacja metody interpolującej podaną funkcję na zadanym przedziale
przy pomocy wyżej zaimplentowach metod oraz węzłów zdefiniowanych w nastepujący sposób:
\[
    x_k = a + k \cdot h, \quad h = (b - a) / n, \quad k = 0, 1, \ldots, n
\]

\subsection{Rozwiązanie}
\subsubsection{Opis metody}
Metoda interpoluje funkcję a następnie wizualizuje na wykres zadaną funkcję, wyznaczony wielomian interpolujący oraz węzły.

\section{Zadanie 5}
\subsection{Cel}
Celem zadania jest przetestowanie zaimplentowanych metod na funkcjach $f_1(x) = e^x$ na przedziale $[0, 1]$ oraz $f_2(x) = x^2 \cdot sin{x}$ na przedziale $[-1, 1]$. 
Badany jest także wpływ stopnia wielomianu na dokładność interpolacji dobierając go ze zbioru ${\{5, 10, 15\}}$.

\subsection{Rozwiązanie}
Zwizualizowano interpolację podanych wyżej funkcji oraz zestawiono wyniki wraz z teoretycznymi przewidywaniami błędu interpolacji.

\subsection{Wyniki}
Wielomiany bardzo dobrze interpolują zadane funkcje $f_1$ oraz $f_2$. Jest to zgodne z teoretyczny przewidywaniami, gdyż błąd interpolacji wyraża się wzorem:
\[
    f(x) - p(x) = \frac{1}{(n+1)!}f^{(n+1)}(\zeta_x)\prod_{i=0}^{n}(x - x_i),
\]
\[
    |f_1(x) - p_1(x)| \leq \frac{1}{6!} \cdot e^1 \cdot \prod_{i=0}^{5}|\frac{1}{2} - x_i| \approx 2 \cdot 10^{-6}
\]
\[
    |f_2(x) - p_2(x)| \leq \frac{1}{6!} \cdot (12 \cdot x \cdot cos(x) - (-30 + x^2) \cdot sin(x)) \prod_{i=0}^{5}|0 - x_i| \approx 5 \cdot 10^{-5}
\]
gdzie $x \in [a, b]$ oraz $\zeta_x \in (a, b)$.


\section{Zadanie 6}
\subsection{Cel}
Celem zadania jest przetestowanie zaimplentowanych metod na funkcjach $f_3(x) = |x|$ na przedziale $[-1, 1]$ oraz $f_4(x) = \frac{1}{1 + x^2}$ na przedziale $[-5, 5]$.
Badany jest także wpływ stopnia wielomianu na dokładność interpolacji dobierając go ze zbioru ${\{5, 10, 15\}}$.

\subsection{Rozwiązanie}
Zwizualizowano interpolację podanych wyżej funkcji oraz zestawiono wyniki wraz z teoretycznymi przewidywaniami błędu interpolacji.

\subsection{Wyniki}
W przypadku funkcji $f_3$ wielomian interpolacyjny gorzej interpoluje zadaną funkcję. Wzrost stopnia wielomianu powoduje lepszą dokładność interpolacji w okolicach zera ale pogarsza ją dla argumentów bliżej końców przedziałów. Powodem takiego zachowania może być fakt, że $f_3$ nie jest funkcją różniczkowalną w przeciwieństwie do wielomianu interpolującego.

W przypadku funkcji $f_4$ wielomian interpolacyjny także gorzej interpoluje zadaną funkcję. Wzrost stopnia wielomianu powoduje lepszą dokładność w okolicach środka układu współrzędnych jednak znacząco pogarsza dokładność dla argumentów bliższych końców zadanego przedziału. 
Jest to efekt Rungego, który polega na pogorszeniu interpolacji pomimo powiększenia stopnia wielomianu interpolującego. Powodem takiego zachowania jest zły dobór węzłów - tj. dobór węzłów w taki sposób aby odległości między nimi były stałe. Dobrowadza to do wniosku, że nie zawsze zwiększenie stopnia wielomianu poprawia dokładność interpolacji. Warto rozważyć inny dobór węzłów.

\newpage

\def \nArray{5, 10, 15}

\foreach \n in \nArray {
    \begin{figure}[h!]
        \centering
        \includegraphics[height=7cm]{../plots/e^x; n=\n.png}
    \end{figure}
}

\newpage 

\foreach \n in \nArray {
    \begin{figure}[h!]
        \centering
        \includegraphics[height=7cm]{../plots/x^2sinx; n=\n.png}
    \end{figure}
}

\newpage

\foreach \n in \nArray {
    \begin{figure}[h!]
        \centering
        \includegraphics[height=7cm]{../plots/|x|; n=\n.png}
    \end{figure}
}

\newpage 

\foreach \n in \nArray {
    \begin{figure}[h!]
        \centering
        \includegraphics[height=7cm]{../plots/frac{1}{1 + x^2}; n=\n.png}
    \end{figure}
}

\end{document}
